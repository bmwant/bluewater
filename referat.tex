\documentclass[14pt]{article}
\usepackage{graphicx}
\usepackage{times}
\usepackage{geometry}
\usepackage{amssymb}
\usepackage{amsmath}
\usepackage{multirow}
\usepackage{tabularx}
\usepackage{blindtext}
\usepackage{natbib}
\geometry{
	a4paper,
	left=20mm,
	top=20mm,
	right=15mm,
	bottom=10mm,
}
\usepackage[utf8]{inputenc}
\usepackage[english,russian,ukrainian]{babel}

\begin{document}

\title{Метод автоматизованої класифікації текстових даних на основі гібридних моделей}


\textbf{Актуальність теми}

\textbf{Об'єктом дослідження} є процес побудови алгоритму універсальної прогностичної моделі.

\textbf{Предметом дослідження} є методи побудови прогностичних моделей та алгоритми класифікації даних.

\textbf{Мета рооти}: створення нового алгоритму побудови прогностичної моделі, що буде демонструвати точність передбачення не меншу, ніж аналогічні моделі для схожого роду вхідних даних, та мати просту реалізацію.

\textbf{Методи дослідження}. В роботі використовуються методи збору даних, методи класифікації текстових даних та статистичні методи.

\textbf{Наукова новизна} роботи полягає в наступному:

\begin{enumerate}
	\item Запропоновано підхід, результатом якого є універсальна прогностична модель, що дає змогу абстрагуватися від конкретних реалізацій і використовувати її для тих самих даних з аналогічними показниками точності та кращими показниками швидкодії.
	\item Наведено процес перетворення будь-якої прогностичної моделі чи деякої композиції моделей для перетворення в універсальну модель.
	\item Підтверджено значно більші показники швидкодії моделі, розробленої за допомогою даного підходу.
\begin{enumerate}
На даному етапі роботи отриманих даних досить для того, щоб почати використовувати даний підхід для роботи з реальними даними та заміною існуючих алгоритмів.

\textbf{Практична цінність} роботи полягає в наступному:
\textbf{Апробація роботи} роботи полягає в наступному:

\textbf{Структура та обсяг роботи}. Магістерська дисертація складається з вступу, п'яти розділів, висновків та додатків.

\underline{У вступі} надано загальну характеристику роботи, виконано оцінку поточного стану проблеми, обґрунтовано актуальність напрямку досліджень.

\underline{У першому розділі} розглянуто теоретичні відомості, існуючі алгоритми класифікації текстових даних, наведено математичні основи, що використовуються для побудови моделей. Розглянуті загальні підходи до автоматизованої класифікації текстових даних та поширені алгоритми, що застосовуються в даній галузі. Основну увагу приділено всьому процесу обробки даних: від їх початкового збору до безпосереднього застосування прогностичної моделі.

\underline{У другому розділі} розглянуто платформу, яка надає можливості для спрощення операцій рефакторингу; проведено огляд типів індексів, які дозволяють пришвидшити операцію пошуку; запропоновано метод рефакторингу структури бази даних для підвищення швидкодії обробки запитів вибірки; описано архітектуру компонентів та технології, обрані для реалізації проекту; наведено вхідні дані та результати роботи методу, надано відповідні ілюстративні матеріали; зазначено перелік та детальний опис кроків методу.

\underline{У третьому розділі} запропоновано засоби реалізації для кожного з етапів методу; наведено огляд архітектурних підходів до організації програмного забезпечення; обґрунтувано вибір мікросервісної архітектури; запропоновано структуру та особливості реалізації кожного з мікросервісів, наведено відповідні графічні матеріали, що ілюструють взаємодію елементів системи.

\underline{У четвертому розділі} наведено результати роботи алгоритму, підтверджено на практиці гіпотезу про те, що застосування розробленого алгоритму надає виграш у швидкодії; отримано підтвердження того, що використання однорідних інструкцій дозволяє зменшити витрати ресурсів процесора; здійснено порівняння точності та швидкості роботи з існуючими алгоритмами; зроблено висновок щодо можливості застосування даного підходу для використання з різними алгоритмами та вхідними даними для вирішення задачі класифікації; запропоновано шляхи покращення та вектори розвитку для подальшої роботи.

\underline{У п'ятому розділі} подано аналіз програмного продукту, його оцінку та перспективи для виходу на ринок. Наведені слабкі та сильні сторони проекту, порівняння з аналогами та конкурентноспроможність. Проведено оцінку розміру необхідних інвестицій, обсягу ресурсів, що потрібно залучити та показників прибутку за умови подальшої комерціалізації проекту.

\underline{У висновках} проаналізовано отримані результати роботи.

У додатках наведено фрагменти програмної реалізації запропонованого способу та копії графічних матеріалів.

Робота виконана на  80 аркушах, містить 2 додатки та посилання на список використаних літературних джерел з 30 найменувань. У роботі наведено 14 рисунків та 4 таблиці.

\textbf{Ключові слова}: класифікація, прогностичні моделі, апроксимація моделі, датасет, машинне навчання.


\end{document}