\section{Висновки}

В дисертаційній роботі запропонований новий метод створення прогностичної моделі, який знайшов відображення в алгоритмі побудови такої моделі та програмній реалізації описаного алгоритму. В рамках розробки методу були створені допоміжні компоненти, що здійснюють попередню обробку, трансформацію, збір відомостей та візуалізацію вхідних даних, а також додаток, що дозволяє використовувати побудовану модель для прогнозування майбутніх значень досліджуваної величини. Додаток надає можливість багатократного використання і повторний запуск на необмеженій кількості нових даних.

Описаний підхід до створення моделі дозволяє отримати високі показники та обмежитися мінімально необхідними ресурсами для запуску на вхідних даних теоретично необмежено розміру з мінімальними втратами точності. Створення такої моделі вимагає додаткового кроку побудови з уже існуючої моделі, але такі затрати є цілком виправданими. За необхідності повторного навчання моделі здійснюється однократна перебудова базової моделі без необхідності тренування всіх моделей, що були побудовані під час стоворення гібридної моделі. Жадібний підхід в цьому випадку є цілком обгрунтований і використаний з метою оптимізації витрат часу на перебудову і підтримки гнучкості, аналогічної до базової моделі. Простота реалізації даного підходу дає змогу покращити та уніфікувати процес побудови прогностичних моделей і їх наступне використання як звичайним науковцям з обробки даних, так і комплексним системам, що містять архітектуру різного рівня складності.

Результати роботи також мають і практичне значення, адже запропонований метод може бути застосований до будь-якого наявного алгоритму класифікації, тому може бути використаний в різноманітних сферах з широким переліком можливих вхідних даних.
Відкритість внутрішньої реалізації дозволяє розробникам модифікувати та змінювати код відповідно до власних потреб, а також додавати нові можливості чи накладати додаткові обмеження до уже існуючого коду. Простота алгоритму кінцевої гібридної моделі дозволяє знизити поріг входження для розробників, що працюють з класифікацією даних, але не мають відповідної математичної освіти або ж обмежені у часі для того, щоб вдаватися в подробиці реалізації моделей та алгоритмів, на яких вони базуються. Розроблена модель є універсальною як з точки зору відсутності строгих обмежень на вхідні дані (до тих пір, доки базові алгоритми не містять таких обмежень), так і з точки зору можливості апроксимувати будь-яку модель незалежно від того, який алгоритм лежить в її основі. 

Базуючись на дослідженні існуючих алгоритмів класифікації аргументовано доцільність розробки не чергового алгоритму класифікації, який буде показувати найкращі результати для визначеного класу задач, а використання найкращого методу для конкретного вхідного набору даних. На основі моделі, побудованої за найкращим у конкретному випадку алгоритмом класифікації, здійснюється створення гібридної моделі, що є апроксимацією даної базової моделі. Саме такий підхід дає змогу поєднати точність передбачення найкращої моделі та покращити швидкодію в декілька разів, оскільки алгоритм роботи гібридної моделі зводиться до примітивного набору порівнянь, які досить ефективно виконуються на сучасних процесорах.

Доведено ефективність побудованої моделі та здійснено порівняння з існуючими алгоритмами класифікації. Перевірка проводилася як на множині різних наборів вхідних даних, що відрізняються за кількість характеристичних ознак, кількістю класів, діапазоном допустимих значень записів, розміром, так і з різними класами алгоритмів, кожен з яких показує найкращі результати лише для окресленого кола проблем. Була здійснена валідація і обрахунок декількох критеріїв за різними метриками для перевірки точності передбачення гібридної моделі та дотримання результатів у межах допустимої похибки від значень еталонної моделі.

Розроблено програмні засоби, які реалізують компоненти для повноцінного використання запропонованого методу на реальних даних, а саме: модуль для збору вхідних даних; модуль для попередньої обробки та трансформації вхідних даних; модуль для тренування моделей, їх оцінки та побудови гібридної моделі на основі вибору кращої моделі за обраним критерієм. Отримані результати експериментальних досліджень розробленого продукту підтвердили теоретичні положення та припущення даної дисертаційної роботи, що дозволило підтвердити ефективність даного методу, а також можливість його використання в якості повноцінної альтернативи існуючим аналогам.

Отже, результати, отримані в межах даної дисертаційної роботи, дають змогу підтвердити ефективність та унікальність запропонованого методу, а також показують доцільність використання розробленого продукту у прикладних ситуаціях та можливість його конкурентноспроможного виходу на ринок.
