\section{Стартап}
Для впровадження продукту в якості стартапу потрібно виділити основну інноваційну складову, яка і буде позиціонувати продукт на ринку та визначати його нішу серед конкурентів. Такою складовою є компонент, що безпосередньо відповідає за побудову моделі та здійснення прогнозування. Саме тому надалі порівняння і критерії будуть виділені лише для цього окремого компоненту.

\begin{table}[H]
\fontsize{12pt}{12pt}\selectfont
    \begin{tabularx}{\textwidth}{|X|X|X|}
    \hline
    Зміст ідеї & Напрямки застовування & Вигоди для користувача \\ \hline
    \multirow{3}{5cm}{Система побудови універсальних прогностичних моделей та метод для додавання  нових алгоритмів в дане рішення} & 1. Використання науковцями та спеціалістами з аналізу даних для підвищення їх ефективності та продуктивності роботи загалом & Зручний та зрозумілий вихідний код дозволить працювати значно ефективніше, тим самим зосереджуючись на прикладних задачах, замість деталей реалізації \\ \cline{2-3}
    & 2. Узагальнення алгоритмів для роботи з різними типами даних & Універсальність моделі дозволить не перемикати контексти під час роботи з різними типами даних, використовуючи однаковий підхід для вхідної інформації \\ \cline{2-3}
    & 3. Отримання кращих результатів передбачень для даних, що змінюються з часом & Допомога під час роботи з величинами, що залежать від часу: курси валют, показники біржі, зміни клімату \\
    \hline
    \end{tabularx}
\caption{Опис ідеї стартап-проекту} \label{tab:stab_0}
\end{table}

Маркетингова діяльність повинна починатися з дослідження макросередовища, вивчення ринку в цілому. Основні категорії, які виділяють для такого роду досліджень: показники виробництва, які характеризують пропозицію товарів; показники, що характеризують попит на товари; показники які характеризують ринкові ціни. Дослідження дало змогу сформувати дані показники в стислій формі таблиці, що відображено на табл. \ref{tab:stab_1}.

\begin{table}[H]
\fontsize{12pt}{12pt}\selectfont
	\begin{tabularx}{\textwidth}{|c|X|X|}
    \hline
    № п/п & Показники стану ринку (найменування) & Характеристика \\ \hline
    1 & Кількість головних гравців, од & 1 \\ \hline
    2 & Загальний обсяг продаж, грн/ум. од & 914 218 млн грн \\ \hline
    3 & Динаміка ринку (якісна оцінка) & Спадає \\ \hline
    4 & Наявність обмежень для входу (вказати характер обмежень) & Висока доля невизначеності, відсутність попереднього досвіду та необхідних статистичних даних \\ \hline
    5 & Специфічні вимоги до стандартизації та сертифікації & - \\ \hline
    6 & Середня норма рентабельності в галузі (або по ринку), \% & 18-20\% \\
    \hline
    \end{tabularx}
\caption{Попередня характеристика потенційного ринку стартап-проекту} \label{tab:stab_1}
\end{table}

Ринок є доволі привабливим для входження: пристойна середня норма рентабельності, що трохи вища за середній банківський відсоток на вклади у гривні, а спадання ринку потенційно відкриває його для нестандартних інноваційних рішень, оскільки існує висока необхідність в розробці методу для побудови універсальної прогрностичної моделі.

\begin{table}[H]
\fontsize{12pt}{12pt}\selectfont
	\begin{tabularx}{\textwidth}{|c|X|X|X|X|}
    \hline
    № п/п & Потреба, що формує ринок & Цільова аудиторія (цільові сегменти ринку) & Відмінності у поведінці різних потенційних цільових груп клієнтів & Вимоги споживачів до товару \\ \hline
    1 & Необхідність для інвесторів знайти перспективний метод для вкладень & Люди, які мають фінансову можливість та зацікавленість робоити інвестиції у інноваційні проекти & Люди, які мають фінансову можливість та зацікавленість робити інвестиції у інноваційні проекти мають на меті збільшення свого капіталу, підвищення свого іміджу, а також долучитися до новітніх технологій, щоб бути у тренді & Необхідно розробити методику оцінювання та рекомендації, які б з високою ймовірністю розраховували потенційні необхідні інвестиції та шляхи попередження ключиових ризиків \\ \hline
    2 & Необхідність команди для побудови цього & Активні люди, які бажають втілити у життя свій проект & Необхідність проаналізувати всі ключові фактори, щоб визначити, чи доцільно реалізовувати проект та чи вдасться залучити спонсорів & Високоточний метод оцінки побудованої прогностичної моделі, щоб визначити доцільність реалізації даного проекту \\
    \hline
    \end{tabularx}
\caption{Характеристика потенційних клієнтів стаптап-проекту} \label{tab:stab_2}
\end{table}

Для того, щоб успішно виживати в довгостроковій перспективі, підприємство повинно вміти передбачати, які саме труднощі можуть виникнути на його шляху в майбутньому і які нові можливості можуть відкритися для нього. Тому аналіз повинен враховувати влив факторів на окремі складові продукту та підприємства в цілому.

\begin{table}[H]
\fontsize{12pt}{12pt}\selectfont
	\begin{tabularx}{\textwidth}{|c|X|X|X|}
    \hline
    № п/п & Фактор & Зміст загрози & Можлива реакція компанії \\ \hline
    1 & Попит & Не вдасться розробити унікальний метод, який би можна було застосовувати для будь-яких алгоритмів та адаптувати для роботи з різними типами даних & Розробка максимально універсального методу \\ \hline 
    2 & Конкуренція & Можливість появи конкурентів з дуже схожими функціями, їх вихід на ринок раніше за нас & Доопрацювання якості розроблюваного методу з фокусом на зручність та простоту використання, розробка нових властивостей, яких немає у конкурента. Розгляд можливості об'єднання компаній для подальної спільної роботи. \\ \hline 
    3 & Економічні & Зменшення доходу інвесторів, що призведе до зменшення кількості інвестицій & Моніторинг економічної ситуації у країні, пошук закордонних користувачів та адаптація для світового ринку \\
    \hline
    \end{tabularx}
\caption{Фактори загроз} \label{tab:stab_3}
\end{table}

Ринкові можливості - це сприятливі обставини, які підприємство може використовувати для отримання переваг. Погіршення позицій конкурентів, різке зростання попиту, поява нових технологій, зростання рівнів доходів населення - це все можливості для проекту, які слід використовувати.

\begin{table}[H]
\fontsize{12pt}{12pt}\selectfont
	\begin{tabularx}{\textwidth}{|c|X|X|X|}
    \hline
    № п/п & Фактор & Зміст можливості & Можлива реакція компанії \\ \hline
    1 & Попит & Унікальність пропонованого функціоналу та додаткових можливостей при умові невисокої конкуренції дозволить захопити велику частку ринку, особливо зацікавивши додатком невеликих інвесторів (бізнес-ангелів) та команди проектів, які не потребують значних інвестицій & Адаптація до ринку, що розширяється, моніторинг новітніх розробок та ризиків, які тіьки нещодавно з'явилися \\ \hline
    2 & Науково-технічні & Поява нових технологій, виникнення нових ринкових умов та факторів, які виявлять значний вплив на розвиток алгоритмів класифікації  & Активне використання використання рішення; у випадку, якщо наше рішення буде адним з перших та матиме суттєві відмінності від аналогів, захист інтелектуальної власності розробників, патентування цієї технології та додання її до інтелектувальних активів проекту \\ \hline
    3 & Соціально-культурні & Велика популярність сфери роботи з даними та їх аналізу & Адаптація системи до розширення ринку, появи нових умов та технологій \\
    \hline
    \end{tabularx}
\caption{Фактори можливостей} \label{tab:stab_3}
\end{table}

Ступеневий аналіз виконується у вигляді таблиці, що описує динаміку розвитку конкурентного середовища та головних гравців ринку. Дані критерії (табл. \ref{tab:stab_4_1}) дозволють оцінити головних конкурентів на ринку за виділеними основними критеріями, що характеризують їх з різних економічних точок зору.

\begin{table}[H]
\fontsize{12pt}{12pt}\selectfont
	\begin{tabularx}{\textwidth}{|X|X|X|}
    \hline
    Особливості конкурентного середовища & В чому проявляється дана характеристика & Вплив на діяльність підприємства (можливі дії компанії, щоб бути конкурентноспроможною) \\ \hline
    1. Тип конкуренції - чиста конкуренція & Велика кількість існуючих методів та алгоритмів, частина з яких є запатентованою інтелектувальною власністю & Звертати увагу на якість та універсальність методу \\ \hline 
    2. За рівнем конкурентної боротьби - національний & Продукт не буде прив'язуватися до географічних показників & Акцент в рекламі на потреби жителів великих міст (столиці), таргетування на науковців та молодих дослідників, а також на високозабезпечених людей - потенційних інвесторів \\ \hline 
    3. За галузевою ознакою - внутрішньогалузева & Конкуренцію складають подібні методики розробки прогностичних моделей & Акцентувати увагу на незвичайність подачі послуг, а також зручність у використанні та надійність, яку вони забезпечують \\
    \hline
    \end{tabularx}
\caption{Ступеневий аналіз конкуренції на ринку} \label{tab:stab_4_1}
\end{table}

\begin{table}[H]
\fontsize{12pt}{12pt}\selectfont
    \begin{tabularx}{\textwidth}{|X|X|X|}
    \hline
    \textit{Особливості конкурентного середовища} & \textit{В чому проявляється дана характеристика} & \textit{Вплив на діяльність підприємства (можливі дії компанії, щоб бути конкурентноспроможною)} \\ \hline
    4. Конкуренція за видами товарів - між бажаннями & Потенційні клієнти роблять вибір між звичними методами побудови моделей (яких дуже велика кількість) і відчувають складність у виборі найбільш доцільного методу & Чітко зрозуміти потреби та бажання кожної з груп цільової аудиторії та розробляти гнучку систему, яка задовольнятиме потреби всіх груп користувачів \\ \hline 
    5. За характером конкурентних переваг - нецінова & Акцент знаходиться на унікальності та якості послуг, що надаються, а також на перевагах, які отримує клієнт під час використання наших послуг & Робота над покращення методики побудови прогностичних моделей та підвищенням її універсальності \\ \hline 
    6. За інтенсивністю - не марочна & Продається втілення ідеї, а не певний бренд & Просування ідеї у соціальних мережах \\
    \hline
    \end{tabularx}
\caption{Ступеневий аналіз конкуренції на ринку (продовження)} \label{tab:stab_4_2}
\end{table}

Особливості конкурентного середовища дають змогу краще зрозуміти, з якого плану умовами та якого роду аналогами доведеться зіткнутися продукту під час боротьби за частку ринку.

\begin{table}[H]
\fontsize{12pt}{12pt}\selectfont
	\begin{tabularx}{\textwidth}{|X|X|X|X|X|X|}
    \hline
    Складові аналізу & Прямі конкуренти в галузі & Потенційні конкуренти & Постачальники & Клієнти & Товари-замінники \\ \hline
     & Прямих конкурентів немає, непрямі - різноманітні методи побудови прогностичних моделей & Нові розробки у галузі & Інвестори диктують умови розвитку ринку: ключова умова - проект повинен бути потрібним користувачам та приносити користь & Кількість зацікавлених клієнтів, рівень зацікавленості в такому типі послуг & Поява схожих дешевших або якісніших продуктів-конкурентів \\ \hline
    Висновки & Прямих конкурентів немає & - можливості входу в ринок присутні, необхідно вирішити проблему пошуку та адаптації статистичних даних - необхідність розробки універсального методу, який може бути використаний як інвесторами, так і командою проекту & Успіх нашого проекту залежить від рівня довіри інвесторів та команд проекту до новітнього методу побудови прогрностичних моделей & Клієнти формують попит на таку послугу & Універсальних методів, які могли б замінити запропонований проект немає \\
    \hline
    \end{tabularx}
\caption{Аналіз конкуренції в галузі за М. Портером} \label{tab:stab_5}
\end{table}

Розроблена Майклом Портером методика для аналізу галузей і вироблення стратегії бізнесу дозволяє оцінити привабливість ведення бізнесу в конкретній галузі, виділяючи п'ять сил, які визначають рівень конкуренції. Привабливість в даному контексті має на увазі рентабельність галузі, тобто найпривабливішою є галузь, що наближається до досконалої конкуренції (табл. \ref{tab:stab_5}). Виділяють п'ять основних сил: аналіз загрози появи продуктів-замінників; аналіз загрози появи нових гравців; аналіз ринкової влади постачальників; аналіз ринкової влади споживачів; аналіз рівня конкурентної боротьби.

\begin{table}[H]
\fontsize{12pt}{12pt}\selectfont
	\begin{tabularx}{\textwidth}{|l|X|X|}
    \hline
    № п/п & Фактор конкурентноспроможності & Обгрунтування (наведення чинників, що роблять фактор для порівнянння конкурентних проектів значущим) \\ \hline
    1 & Фактор часу & Ідея є частково новою, для перейняття ідеї та втілення її у життя потенційним конкурентам знадобиться час \\ \hline
    2 & Фактор новизни товару & Початковий успіх продукту очікується через його новизну та інтерес цільової аудиторії до нових інноваційних рішень \\ \hline
    3 & Фактор якості послуг та надання інформації & Науковці та експерти з обробки даних потребують універсальний метод побудови прогностичних моделей \\
    \hline
    \end{tabularx}
\caption{Обґрунтування факторів конкурентноспроможності} \label{tab:stab_6}
\end{table}

Аналіз внутрішніх сильних і слабких сторін рекомендується проводити як порівняльний аналіз, причому головний напрям уваги має спрямовуватися на конкурентноспроможність продукту. Критерії, які повинні оцінюватися охоплюють цілу низку показників, до основних груп яких входять: прибутковість; репутація; продуктивність; асортимент; впровадження іновацій.

\begin{table}[H]
\fontsize{12pt}{12pt}\selectfont
	\begin{tabularx}{\textwidth}{|l|X|X|X|X|X|X|X|}
    \hline
    № п/п & Фактор конкурентноспроможності & Бали 1-20 Рейтинг товарів-конкурентів у порівнянні з іншими методами оцінювання & 2 & 3 & 4 & 5 & 6 \\ \hline
    1 & Фактор часу & 15 & & & + & & \\ \hline
    2 & Фактор новизни товару & 20 & & + & & & \\ \hline
    3 & Фактор якості послуг та надання інформації & 17 & & + & & & \\
    \hline
    \end{tabularx}
\caption{Порівняльний аналіз сильних та слабких сторін методу} \label{tab:stab_7}
\end{table}

У процесі стратегічного планування для представлення і структуризації зань про поточну ситуацію і тенденції  краще всього скористатися \textit{SWOT}-аналізом, що полягає в розділенні чинників і явищ на чотири категорії: сильні сторони, слабкі сторони проекту; можливості, що відкриваються при його реалізації; загрози, пов'язані з його здійсненням (табл. \ref{tab:stab_8}). Інформація по кожному з напрямків може оцінюватися також і по кількісних величинам, на основі яких за допомогою функцій корисності обчислюється потенціал досліджуваного об'єкта по кожному з напрямків.

\begin{table}[H]
\fontsize{12pt}{12pt}\selectfont
	\begin{tabularx}{\textwidth}{|X|X|}
    \hline
    Сильні сторони: 
    Якість послуг, що надаються
    Новизна послуг
    Можливість використання як інвесторами, і командою з розробки
     & Слабкі сторони:
     Відсутність статистичних даних та попереднього досвіду в реалізації подібних рішень \\ \hline
    Можливості: 
    Створення нової ринкової ніші
    Потреба у ефективному та компактному методі створення прогностичних моделей
    Необхідність закладати у бюджет можливі ризики та зміни ринкових умов
     & Загрози:
     Різка зміна ринку, поява нових стартапів, економічна криза \\
    \hline
    \end{tabularx}
\caption{SWOT-аналіз стартап-проекту} \label{tab:stab_8}
\end{table}

Результатом аналізу є сформоване узагальнення інформаційного потенціалу та побудова ефективних рішень, що стосуються відповідної реакції суб'єкта відповідно до сигналу зовнішнього середовища (табл. \ref{tab:stab_9}). Сильні сторони проекту покликані забезпечити його прискорене просування до досягнення стратегічних цілей.

\begin{table}[H]
\fontsize{12pt}{12pt}\selectfont
	\begin{tabularx}{\textwidth}{|l|X|X|X|}
    \hline
    № п/п & Альтернатива (орієнтовний комплекс заходів) ринкової поведінки & Ймовірність отримання ресурсів & Строки реалізації \\ \hline
    1 & - Ціль: отримання прибутку в короткостроковій перспективі
    - Конкуренція: цінова та партнерська (пропонуємо свої нові послуги розповсюдження інформації про партнерів - рекламні послуги)
    - Взаємодія з фірмами: активна боротьба за долю ринку, що належить конкурентам & В короткостроковому плані - велика
    В довгостроковому плані - значний ризик втратити долю ринку, якщо займатися лише ціновою конкуренцією & 8-12 місяців після запуску проекту \\ \hline
    2 & - Ціль: захоплення частини ринку, підтримання її розміру та поступове нарощення об'ємів
    - Конкуренція: нецінова (акцент на тому, що пропонуємо інноваційні послуги)
    - Взаємодія з конкурентами: співпраця, активний моніторинг їх діяльності, при можливій появі реальних конкурентів можна запропонувати злиття компаній/проектів & Висока ймовірність отримання ресурсів та утримання їх протягом довгого проміжку часу. Більш ймовірний розвиток компанії та постійне покращення продукту & 8-12 місяців після запуску проекту - для отримання перших фінансових надходжень від розповсюдження інформації про акції магазинів-партнерів, та їх реклама. Далі фінансові надходження прогнозовано регулярними \\
    \hline
    \end{tabularx}
\caption{Альтернативи ринкового впровадження стартап-проекту} \label{tab:stab_9}
\end{table}

Стратегічні альтернативи є обов'язковою умовою реалізації стратегії зростання та забезпечується наступною їхньою конкретизаціює та розробкою функціональних і ресурсних субстратегій. В якості альтернативної стратегії обрано \textit{другу} як таку, що має на увазі довше життя проекту.

\begin{table}[H]
\fontsize{12pt}{12pt}\selectfont
	\begin{tabularx}{\textwidth}{|l|X|X|X|X|X|}
    \hline
    № п/п & Опис профілю цільової групи потенційних клієнтів & Готовність споживачів сприйняти продукт & Орієнтовний попит в межах цільової групи (сегменту) & Інтенсивність конкуренції в сегменті & Простота входу у сегмент \\ \hline
    1 & Високозабезпечені люди, які зацікавлені у пошуку перспективних проектів для інвестування & Споживачі слідкують за найновітнішими технологіями, бажають бути в тренді та готові сприйняти новий продукт & Потенційно високий, інвестори хочуть бути впевненими у доцільності своїх інвестицій та подальшому отриманні прибутку & Практично відсутня & При наявності достойної та доручної реклами - досить просто \\ \hline
    2 & Ініціативні люди та науковці, які мають хорошу ідею в схожій сфері та хочуть втілити її у життя & Споживачі готові сприйняти продукт, так як зацікавлені у глибинному аналізі ситуації & Високий попит & Практично відсутня & При наявності достойної та доречної реклами - досить просто \\
    \hline
    \end{tabularx}
\caption{Вибір цільових груп потенційних споживачів} \label{tab:stab_10}
\end{table}

До цільової групи відноситься група осіб, на задоволення потреб яких спрямовано проект. Сутність визначення об'єкта проекту полягає в тому, щоб обрати саме ту цільову групу, яка найбільше потерпає від проблеми, на розв'язання якої спрямований проект або може найбільше вплинути на характеристики цієї проблеми. В межах одного проекту можуть існувати кілька рівнозначних об'єктів (табл. \ref{tab:stab_10}). До основних критеріїв вибору об'єкта проекту відносяться такі: відповідність об'єкта ідеології проекту; відповідність об'єкта цілям проекту; відповідність об'єкта тим чинникам, які найбільше детермінують кризові ситуації клієнтів проекту; відповідність об'єкта проекту особливостям його внутрішньої структури. Отже, в результаті аналізу в якості цільових груп було обрано \textit{першу} та \textit{другу} групи.

\begin{table}[H]
\fontsize{12pt}{12pt}\selectfont
	\begin{tabularx}{\textwidth}{|l|X|X|X|X|}
    \hline
    № п/п & Обрана альтернатива розвитку проекту & Стратегія охоплення ринку & Ключові конкурентноспроможні позиції відповідно до обраної альтернативи & Базова стратегія розвитку \\ \hline
    1 & Захоплення, підтримання та захист частки ринку & Стратегія концентрованого маркетингу & - Новизна послуг
    - Доступність продукту
    - Простота в користуванні продуктом
    - Додаткові зручні аспекти, які враховуються під час розрахунку ефективності та інвестиційної привабливості побудови прогностичних моделей, що вигідно виділяють наш продукт серед конкурентів & Стратегія диференціації \\
    \hline
    \end{tabularx}
\caption{Визначення базової стратегії розвитку} \label{tab:stab_11}
\end{table}

Для стратегічного планування відправним моментом є вибір базової стратегії. Вихідними даними для вибору базової стратегії служать як макроекономічні чинники, так і внутрішні можливості, що визначаються циклом розвитку продукту, а основне завдання, що вирішується при цьому, полягає у забезпеченні узгодженості між цілями і ресурсами (табл. \ref{tab:stab_12}). Як результат спроб досягнення такого узгодження і виникають альтернативні варіанти. Кожному з них відповідає конкретний блок методів із забезпечення обраних напрямків діяльності, удосконалення організаційної структури і системи управління, освоєння нових технологій.

\begin{table}[H]
\fontsize{12pt}{12pt}\selectfont
	\begin{tabularx}{\textwidth}{|l|X|X|X|X|}
    \hline
    № п/п & Чи є проект "першопрохідцем" на ринку & Чи буде компанія шукати нових споживачів, або забирати існуючих у конкурентів & Чи буде компанія копіювати основні характеристики товару конкурента і які? & Стратегія конкурентної поведінки \\ \hline
    1 & Частково & Нові споживачі, частково забиратиме споживачів конкурентів & Частково. Новий метод оцінювання ефективності побудови прогностичних моделей буде агрегувати декілька методик аналізу, що дозволить оцінювати проекти більш точно з використанням більшої кількості факторів, що впливають на проект & Стратегія лідера \\
    \hline
    \end{tabularx}
\caption{Визначення базової стратегії конкурентної поведінки} \label{tab:stab_12}
\end{table}

Конкурента поведінка - це операційна поступова поведінка з метою отримання прибутку в умовах, коли існуючі ринки дозволяють забезпечувати цільовий рівень виробництва й прибутку. Поведінку в конкурентному середовищі необхідно розглядати не тільки з погляду конкурентної активності, ступеня адаптивності та конкурентного статусу, але також з урахуванням таких характеристик, як орієнтація на дії конкурентів, раціональність дій.
\begin{table}[H]
\fontsize{12pt}{12pt}\selectfont
	\begin{tabularx}{\textwidth}{|l|X|X|X|X|}
    \hline
    № п/п & Вимоги до товару цільової аудиторії & Базова стратегія розвитку & Ключові конкурентноспроможні позиції власного стартап-проекту & Вибір асоціацій, які мають сформувати комплексну позицію власного проекту (три ключових) \\ \hline
    1 & Універсальність методу оцінювання з точки зору інвестора & Стратегія диференціації & Врахування всіх аспектів оцінювання проекту з точки зору інвестиційної привабливості & - Ваші гроші ефективно працюють у інноваційному прогресивному проекті \\ \hline
    2 & Універсальність методу оцінювання з точки зору команди проекту & Стратегія диференціації & Врахування всіх аспектів оцінювання проекту з точки зору інвестиційної привабливості та життєздатност іпроекту, доцільність реалізовувати інноваційний проект & - Реальна можливість втілити у життя ідею завдяки глибинному аналізу ключових аспектів та пошуку інвесторів \\ \hline
    3 & Необхідність враховувати ризики проекту, ринкові та економічні умови, що швидко змінюються & Стратегія диференціації & Врахування ключових ризиків та ринкових умов завдяки розробленій системі коефіцієнтів & - Детальний облік ризиків та моніторинг ринкових умов дозволять уникнути передчасного закриття проекту \\
    \hline
    \end{tabularx}
\caption{Визначення стратегій позиціонування} \label{tab:stab_13}
\end{table}

Користувач віддає перевагу продукту, якість, властивості і характеристики якого постійно поліпшуються, а отже концепція майбутнього продукту повинна враховувати дані критерії, а також переваги перед конкурентами, як наявні, так і такі, що потрібно створити задля підвищення якості свого продукту (табл. \ref{tab:stab_14}).

\begin{table}[H]
\fontsize{12pt}{12pt}\selectfont
	\begin{tabularx}{\textwidth}{|l|X|X|X|}
    \hline
    № п/п & Потреба & Вигода, яку пропонує товар & Ключові переваги перед конкурентами (існуючі або такі, що потрібно створити) \\ \hline
    1 & Універсальний метод оцінювання ефективності та інвестиційної привабливості проекту, який буде корисний як для інвесторів, так і для команди проекту & Методика оцінювання дозволить уникнути передчасного закриття проекту та перевитрат люджету завдяки високоточній оцінці на ранніх етапах проекту & Оцінювання проекту як з точки зори витрат та ефективності їх використання командою стартапу, так і з урахуванням потенційних ризиків, прихованих стратегічних переваг на недоліків. \\
    \hline
    \end{tabularx}
\caption{Визначення ключових переваг концепції потенційного товару} \label{tab:stab_14}
\end{table}

Встановлення нижньої межі ціни на товар або послугу визначається визначається витратами, в той час як верхню межу визначає ринок, спираючись на закони попиту і пропозиції. З програмними продуктами відбувається дещо по-ішному, оскільки нижню межу визначають декілька факторів: витрати на виробництво програмного продукту, якщо продукт створюється власними зусиллями, тобто без залучення чужих інструментальних засобів; упущена вигода, пов'язана з відмовою від самостійних дій на ринку в разі передачі продукту посередникам для подальшого розповсюдження чи зростанням ризику при розголошенні функціонального змісту і можливісті несанкціонованого копіювання та розповсюдження.

\begin{table}[H]
\fontsize{12pt}{12pt}\selectfont
	\begin{tabularx}{\textwidth}{|l|X|X|X|X|}
    \hline
    № п/п & Рівень цін на товари-замінники & Рівень цін на товари-аналоги & Рівень доходів цільової групи споживачів & Верхня та нижня межі встановлення ціни на товар/послугу \\ \hline
    1 & Безкоштовно & Безкоштовно & Більше 10000 грн/місяць & - \\
    \hline
    \end{tabularx}
\caption{Визначення меж встановлення ціни} \label{tab:stab_15}
\end{table}

Для забезпечення ефективної реалізації створеного продукту здійснюють комплекс заходів з розробки маркетингової збутової стратегії. Збут - один з головних елементів маркетингу, ключова роль якого обумовлена такими обставинами: у сфері збуту визначається результат комерційного виробництва; пристосування збутової мережі до запитів споживачів впливає на перемогу у конкурентній боротьбі; збутова мережа продовжує процес виробництва; на стадії збуту чітко вимальовуються смаки, запити і переваги споживачів. Дослідження основних форм і методів збуту спрямоване на пошук перспективних засобів просування товарів від виробника до кінцевого споживача і організацію роздрібної торгівлі на основі всестороннього аналізу і оцінки ефективності використовуваних каналів і способів розподілу і збуту (табл. \ref{tab:stab_16}).

\begin{table}[H]
\fontsize{12pt}{12pt}\selectfont
	\begin{tabularx}{\textwidth}{|l|X|X|X|X|}
    \hline
    № п/п & Специфіка закупівельної поведінки цільових клієнтів & Функції збуту, які має виконувати постачальник товару & Глибина каналу збуту & Оптимальна система збуту \\ \hline
    1 & Купують право на використання методики & Зберігання, сортування, встановлення контакту інформування & Однорівневий & Залучена \\
    \hline
    \end{tabularx}
\caption{Формування системи збуту} \label{tab:stab_16}
\end{table}

Наостанок, важливо виділити маркетингові комунікації - це концепція, згідно з якою компанія ретельно обмірковує і координує роботу своїх численних каналів комунікацї, рекламу в засобах масової інформації, особистий продаж, стимулювання збуту, пропаганду, прямий маркетинг. Це робиться з метою вироблення чіткого, послідовного і переконливого уявлення про компанію і її товари (табл. \ref{tab:stab_17}).

\begin{table}[H]
\fontsize{12pt}{12pt}\selectfont
	\begin{tabularx}{\textwidth}{|l|X|X|X|X|X|}
    \hline
    № п/п & Спеціфіка поведінки цільових клієнтів & Канали комунікацій, якими користуються цільові клієнти & Ключові позиції, обрані для позиціонування & Завдання рекламного повідомлення & Концепція рекламного звернення \\ \hline
    1 & Позитивне відношення до інновацій та швидкий розвиток технологій призводять до появи великої кількості нових методів, росту кількості даних і побудова прогностичних моделей стає більш актуальною & Соціальні мережі (facebook, twitter), тематичні ресурси & Універсальний метод побудови прогностичних моделей & Впевнити клієнта у тому, що метод є унікальним та універсальним & Повідомлення у соціальних мережах, статті на веб-ресурсах, короткі демонстраційні ролики \\
    \hline
    \end{tabularx}
\caption{Концепція маркетингових комунікацій} \label{tab:stab_17}
\end{table}

\newpage
\subsection{Результати дослідження ринку}
Проведений детальний аналіз ринку та перспектив розвитку проекту дав змогу отримати такі результати:
\begin{itemize}  
	\item Існує можливість ринкової комерціалізації проекту, на ринку наявний попит на пропонований продукт.
	\item Ринок відкритий для інновацій, прослідковується позитивна динаміка ринку. 
	\item Рентабельність роботи на ринку вища за прибутковість банківських вкладів, а отже приваблює як інвесторів, так і розробників для роботи над перспективним проектом. 
	\item З огляду на потенційні групи клієнтів існує потенціал та перспектива входу на ринок. 
	\item Істотні бар'єри для входження відсутні.
	\item В якості варіанту для впровадження для ринкової реалізації проекту доцільно обрати довгострокову роботу та утримання клієнтів, роботу над покращенням розробленого методу з використанням багатовимірного статистичного аналізу.
	\item Конкуренція практично відсутня, а конкурентноспроможність самого продукту достатньо висока.
\end{itemize}
Враховуючи описані вище ключові моменти, можна зробити висновок, що подальша імплементація даного проекту є доцільною та обґрунтованою.
