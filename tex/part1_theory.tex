\section{Теоретичні основи е-е}
На відміну від штучно створених мов, наприклад мов програмування чи математичних нотацій,
мови, які ми використовуємо для спілкування, розвивалися з покоління в покоління, постійно
видозмінюючись, а тому досить складно відслідкувати і встановити набір чітких конкретно
визначених правил. Розробка алгоритмів, що дозволяють "розуміти" людські висловлювання
дає змогу покращити велику кількість аспектів взаємодії людини та комп'ютера: передбачення
вводу, розпізнавання тексту, пошук інформації в неструктурованому тексті, переклад з однієї
мови на іншу, аналіз емоційного забарвлення тексту та багато іншого. Створюючи інтерфейси,
що дозволяють людині більш ефективно використовувати комп'ютер, ми прискорюємо
розвиток багатомовного інформаційного суспільства.


Прогностичне моделювання – використання статистичних методів для передбачення деякого цільового значення. Зазвичай, мається на увазі передбачення деякої величини в майбутньому, хоча узагальнено це не грає жодної ролі і може бути застосовано до будь-якого типу невідомої події, незалежно від того, коли вона відбулася.
В багатьох випадках задача зводиться до вибору найкращої моделі, що намагається здогадатися результат на основі набору вхідних даних, наприклад визначення того, чи є деякий лист електронної пошти спамом. Моделі можуть використовувати один чи декілька класифікаторів, щоб визначати приналежність даних до деякої множини. Сам термін прогностичної моделі широко перетинається з поняттями машинного навчання в наукових статтях та в контексті розробки програмного забезпечення. В промисловому середовищі даний термін швидше відноситься до поняття прогностичного аналізу.

The text classification problem \cite{DUMMY:1}

In text classification, we are given a description
$\mathbb{X}$ of a document, where $\mathbb{X}$ is the document space ; and a fixed set
of classes  $\mathbb{C}$ Classes are also called categories or
labels . Typically, the document space  $\mathbb{X}$ is some type of high-dimensional space,
and the classes are human defined for the needs of an application, as in the examples China
and documents that talk about multicore computer chips above.

\subsection{Exploratory data analysis}
Візуалізація для наступних цілей:
* Комунікативна
- представлення даних та ідей
- проінформувати
- підтримати і аргументувати
- вплинути і переконати
* Дослідницька
- вивчити (дослідити) дані
- проаналізувати ситуацію
- визначити наступні кроки
- прийняти рішення стосовно деякого питання

\begin{equation}
    \label{simple_equation}
    \alpha = \sqrt{ \beta }
\end{equation}

\subsection{Класифікація тексту}
Метою класифікації текстів є розподіл документів на групи наперед визначених категорій. *-*


\subsection{Учи матчасть}
Результати показують, що стабільно показують чудові результати для завдань класифікації
текстів, суттєво перевищуючи показники інших методів.

