\section{Програмна реалізація}
Особливості і деталі програмної реалізації

\subsection{Збір та попередня обробка даних}
Для збору даних та формування початкового датасету було створено допоміжний додаток у вигляді веб-ресурсу. Він являє собою веб-сайт, на якому здійснюються опитування серед різних класів респондентів: студентів, випускників та викладачів. Кожен учасник опитування заповнює невелику тематичну анкету, на основі якої формуюється таблиця вхідних даних. 

Для створення ефективної кінцевої моделі розроблюваний алгоритм повинен підповідати таким вимогам:
\begin{itemize}  
	\item відкритий доступ користувача до коду моделі на будь-якій платформонезалежній мові, що дозволить запускати її в довільному середовищі та не опиратися на використання сторонніх бібліотек;
	\item будова моделі та деталі її внутрішньої реалізації повинні бути відкритими, тобто користувач повинен мати змогу переглянути вихідний код і в разі необхідності самостійно відтворити довільний крок та отримати аналогічний результат передбачення для однакового набору вхідних даних;
	\item модель повинна мати точність максимально наближену до точності моделей, що показують найкращі результати для вибраних вхідних даних. Модель повинна мати аналогічні показники щонайменше для 95\% всіх вхідних наборів даних;
	\item виконання коду програми повинно бути швидким (близько 1 мс на рядок вхідних даних).
\end{itemize}
