\subsection{Розділ 4. Стартап}
Таблиця 1. Опис ідеї стартап-проекту

With width specified:
\begin{center}
    \begin{tabular}{ | l | l | l | p{5cm} |}
    \hline
    Day & Min Temp & Max Temp & Summary \\ \hline
    Monday & 11C & 22C & A clear day with lots of sunshine.  
    However, the strong breeze will bring down the temperatures. \\ \hline
    Tuesday & 9C & 19C & Cloudy with rain, across many northern regions. Clear spells 
    across most of Scotland and Northern Ireland, 
    but rain reaching the far northwest. \\ \hline
    Wednesday & 10C & 21C & Rain will still linger for the morning. 
    Conditions will improve by early afternoon and continue 
    throughout the evening. \\
    \hline
    \end{tabular}
\end{center}

\begin{center}
	\begin{tabular}{ | l  | l | l |}
	\hline
	Зміст ідеї & Напрямки застовування & Вигоди для користувача \\ \hline
	\multirow{3}{*}{Система відновлення неякісних зображень} & 1. Покращення зображень для систем відеонагляду &  Отримання більшої кількості \\ \hline
	2. Відновлення домашніх фотознімків & Дає змогу покращити кадру \\ \hline
	3. Покращення якості мрт & Віднайдення життя громадян \\
	\hline
	\end{tabular}
\end{center}

\begin{tabular}{|l|l|l|l|}\hline
  \multirow{10}{*}{numeric literals} & \multirow{5}{*}{integers} & in decimal & \verb|8743| \\ \cline{3-4}
  & & \multirow{2}{*}{in octal} & \verb|0o7464| \\ \cline{4-4}
  & & & \verb|0O103| \\ \cline{3-4}
  & & \multirow{2}{*}{in hexadecimal} & \verb|0x5A0FF| \\ \cline{4-4}
  & & & \verb|0xE0F2| \\ \cline{2-4}
  & \multirow{5}{*}{fractionals} & \multirow{5}{*}{in decimal} & \verb|140.58| \\ \cline{4-4}
  & & & \verb|8.04e7| \\ \cline{4-4}
  & & & \verb|0.347E+12| \\ \cline{4-4}
  & & & \verb|5.47E-12| \\ \cline{4-4}
  & & & \verb|47e22| \\ \cline{1-4}
  \multicolumn{3}{|l|}{\multirow{3}{*}{char literals}} & \verb|'H'| \\ \cline{4-4}
  \multicolumn{3}{|l|}{} & \verb|'\n'| \\ \cline{4-4}          %% here
  \multicolumn{3}{|l|}{} & \verb|'\x65'| \\ \cline{1-4}        %% here
  \multicolumn{3}{|l|}{\multirow{2}{*}{string literals}} & \verb|"bom dia"| \\ \cline{4-4}
  \multicolumn{3}{|l|}{} & \verb|"ouro preto\nmg"| \\ \cline{1-4}          %% here
\end{tabular}

\begin{table}
	\begin{tabularx}{\textwidth}{|c|X|X|}
    \hline
    № п/п & Показники стану ринку (найменування) & Характеристика \\ \hline
    1 & Кількість головних гравців, од & 1 \\ \hline
    2 & Загальний обсяг продаж, грн/ум. од & 914 218 млн грн \\ \hline
    3 & Динаміка ринку (якісна оцінка) & Спадає \\ \hline
    4 & Наявність обмежень для входу (вказати характер обмежень) & Висока доля невизначеності, відсутність попереднього досвіду та необхідних статистичних даних \\ \hline
    5 & Специфічні вимоги до стандартизації та сертифікації & - \\ \hline
    6 & Середня норма рентабельності в галузі (або по ринку), \% & 18-20\% \\
    \hline
    \end{tabularx}
\caption{Попередня характеристика потенційного ринку стартап-проекту} \label{tab:sometab}
\end{table}
Ринок є доволі привабливим для входження: пристойна середня норма рентабельності, що трохи вищ аза середній банківський відсоток на вклади у гривні, а спадання ринку потенційно відкриває його для нестандартних інноваційних рішень, оскільки існує дуже висока необхідність в розробці універсального методу для відновлення зображень.

\begin{table}
	\begin{tabularx}{\textwidth}{|c|X|X|X|X|}
    \hline
    № п/п & Потреба, що формує ринок & Цільова аудиторія (цільові сегменти ринку) & Відмінності у поведінці різних потенційних цільових груп клієнтів & Вимоги споживачів до товару \\ \hline
    1 & Необхідність для інвесторів знайти перспективний метод для вкладень & Люди, які мають фінансову можливість та зацікавленість робоити інвестиції у інноваційні проекти & Люди, які мають фінансову можливість та зацікавленість робити інвестиції у інноваційні проекти мають на меті збільшення свого капіталу, підвищення свого іміджу, а також долучитися до новітніх технологій, щоб бути у тренді & Необхідно розробити методику оцінювання та рекомендації, які б з високою ймовірністю розраховували потенційні необхідні інвестиції та шляхи попередження ключиових ризиків \\ \hline
    2 & Необхідність команди для побудови цього & Активні люди, які бажають втілити у життя свій проект & Необхідність проаналізувати всі ключові фактори, щоб визначити, чи доцільно реалізовувати проект та чи вдасться залучити спонсорів & Високоточний метод оцінки відновлення зображень, щоб визначити доцільність реалізації відновлення зображень \\
    \hline
    \end{tabularx}
\caption{Характеристика потенційних клієнтів стаптап-проекту} \label{tab:sometab}
\end{table}


\begin{table}
	\begin{tabularx}{\textwidth}{|c|X|X|X|}
    \hline
    № п/п & Фактор & Зміст загрози & Можлива реакція компанії \\ \hline
    1 & Попит & Не вдасться розробити унікальний метод, який би можна було застосовувати для будь-яких відновлення зображень & Розробка максимально універсального методу \\ \hline 
    2 & Науково-технічні & Поява нових технологій, виникнення нових ринкових умов та факторів, які дуже сильно впливають на відновлення зображень & Активне використання наввних рішень; у випадку, якщо наше рішення буде одним з перших та матиме суттєві відмінності від аналогів, захист інтелектуальної власності розробників, патентування цієї технології та додання її до інтелектуальних активів проекту \\ \hline 
    3 & Соціально-культурні & Велика популярність відновлення зображень & Адаптація системи до розширення ринку, появи нових умов та технологій \\
    \hline
    \end{tabularx}
\caption{Фактори можливостей} \label{tab:sometab}
\end{table}

\begin{table}
	\begin{tabularx}{\textwidth}{|l|X|X|}
    \hline
    Особливості конкурентного середовища & В чому проявляється дана характеристика & Вплив на діяльність підприємства (можливі дії компанії, щоб бути конкурентноспроможною) \\ \hline
    1. Тип конкуренції - чиста конкуренція & Велика кількість методів відновлення зображень, частина з яких є запатентованою інтелектувальною власністю & Звертати увагу на якість та універсальність методу відновлення зображень \\ \hline 
    2. За рівнем конкурентної боротьби - національний & Відновлення зображень не буде прив'язуватися до географічних показників & Акцент в рекламі на потреби жителів великих міст (столиці), таргетування на науковців та молодих дослідників, а також на високозабезпечених людей - потенційних інвесторів \\ \hline 
    3. За галузевою ознакою - внутрішньогалузева & Конкуренцію складають подібні методики розробки прогностичних моделей & Акцентувати увагу на незвичайність подачі послуг, а також зручність у використанні та надійність, яку вони забезпечують \\ \hline 
    4. Конкуренція за видами товарів - між бажаннями & Потенційні клієнти роблять вибір між звичними методами побудови моделей (яких дуже велика кількість) і відчувають складність у виборі найбільш доцільного методу & Чітко зрозуміти потреби та бажання кожної з груп цільової аудиторії та розробляти гнучку систему, яка задовольнятиме потреби всіх груп користувачів \\ \hline 
    5. За характером конкурентних переваг - нецінова & Акцент знаходиться на унікальності та якості послуг, що надаються, а також на перевагах, які отримує клієнт під час використання наших послуг & Робота над покращення методики побудови прогностичних моделей та підвищенням її універсальності \\ \hline 
    3. За інтенсивністю - не марочна & Продається втілення ідеї, а не певний бренд & Просування ідеї у соціальних мережах \\ \hline 
    \hline
    \end{tabularx}
\caption{Ступеневий аналіз конкуренції на ринку} \label{tab:sometab}
\end{table}

\begin{table}
	\begin{tabularx}{\textwidth}{|X|X|X|X|X|X|}
    \hline
    Складові аналізу & Прямі конкуренти в галузі & Потенційні конкуренти & Постачальники & Клієнти & Товари-замінники \\ \hline
     & Прямих конкурентів немає, непрямі - різноманітні методи побудови прогностичних моделей & Нові розробки у галузі & Інвестори диктують умови розвитку ринку: ключова умова - проект повинен бути потрібним користувачам та приносити користь & Кількість зацікавлених клієнтів, рівень зацікавленості в такому типі послуг & Поява схожих дешевших або якісніших продуктів-конкурентів \\ \hline
    Висновки & Прямих конкурентів немає & - можливості входу в ринок присутні, необхідно вирішити проблему пошуку та адаптації статистичних даних - необхідність розробки універсального методу, який може бути використаний як інвесторами, так і командою проекту & Успіх нашого проекту залежить від рівня довіри інвесторів та команд проекту до новітнього методу побудови прогрностичних моделей & Клієнти формують попит на таку послугу & Універсальних методів, які могли б замінити запропонований проект немає \\
    \hline
    \end{tabularx}
\caption{Аналіз конкуренції в галузі за М. Портером} \label{tab:sometab}
\end{table}

\begin{table}
	\begin{tabularx}{\textwidth}{|l|X|X|}
    \hline
    № п/п & Фактор конкурентноспроможності & Обгрунтування (наведення чинників, що роблять фактор для порівнянння конкурентних проектів значущим) \\ \hline
    1 & Фактор часу & Ідея є частково новою, для перейняття ідеї та втілення її у життя потенційним конкурентам знадобиться час \\ \hline
    2 & Фактор новизни товару & Початковий успіх продукту очікується через його новизну та інтерес цільової аудиторії до нових інноваційних рішень \\ \hline
    3 & Фактор якості послуг та надання інформації & Науковці та експерти з обробки даних потребують універсальний метод побудови прогностичних моделей \\
    \hline
    \end{tabularx}
\caption{Обґрунтування факторів конкурентноспроможності} \label{tab:sometab}
\end{table}

\begin{table}
	\begin{tabularx}{\textwidth}{|l|X|X|}
    \hline
    № п/п & Фактор конкурентноспроможності & Бали 1-20 Рейтинг товарів-конкурентів у порівнянні з іншими методами оцінювання & 2 & 3 & 4 & 5 & 6 \\ \hline
    1 & Фактор часу & 15 & & & + & & \\ \hline
    2 & Фактор новизни товару & 20 & & + & & & \\ \hline
    3 & Фактор якості послуг та надання інформації & 17 & & + & & & \\
    \hline
    \end{tabularx}
\caption{Порівняльний аналіз сильних та слабких сторін методу} \label{tab:sometab}
\end{table}

\begin{table}
	\begin{tabularx}{\textwidth}{|l|l|}
    \hline
    Сильні сторони: 
    Якість послуг, що надаються
    Новизна послуг
    Можливість використання як інвесторами, і командою з розробки
     & Слабкі сторони:
     Відсутність статистичних даних та попереднього досвіду в реалізації подібних рішень \\ \hline
    Можливості: 
    Створення нової ринкової ніші
    Потреба у ефективному та компактному методі створення прогностичних моделей
    Необхідність закладати у бюджет можливі ризики та зміни ринкових умов
     & Загрози:
     Різка зміна ринку, поява нових стартапів, економічна криза \\
    \hline
    \end{tabularx}
\caption{SWOT-аналіз стартап-проекту} \label{tab:sometab}
\end{table}

\begin{table}
	\begin{tabularx}{\textwidth}{|l|X|X|X|}
    \hline
    № п/п & Альтернатива (орієнтовний комплекс заходів) ринкової поведінки & Ймовірність отримання ресурсів & Строки реалізації \\ \hline
    1 & - Ціль: отримання прибутку в короткостроковій перспективі
    - Конкуренція: цінова та партнерська (пропонуємо свої нові послуги розповсюдження інформації про партнерів - рекламні послуги)
    - Взаємодія з фірмами: активна боротьба за долю ринку, що належить конкурентам & В короткостроковому плані - велика
    В довгостроковому плані - значний ризик втратити долю ринку, якщо займатися лише ціновою конкуренцією & 8-12 місяців після запуску проекту \\ \hline
    2 & - Ціль: захоплення частини ринку, підтримання її розміру та поступове нарощення об'ємів
    - Конкуренція: нецінова (акцент на тому, що пропонуємо інноваційні послуги)
    - Взаємодія з конкурентами: співпраця, активний моніторинг їх діяльності, при можливій появі реальних конкурентів можна запропонувати злиття компаній/проектів & Висока ймовірність отримання ресурсів та утримання їх протягом довгого проміжку часу. Більш ймовірний розвиток компанії та постійне покращення продукту & 8-12 місяців після запуску проекту - для отримання перших фінансових надходжень від розповсюдження інформації про акції магазинів-партнерів, та їх реклама. Далі фінансові надходження прогнозовано регулярними \\
    \hline
    \end{tabularx}
\caption{Альтернативи ринкового впровадження стартап-проекту} \label{tab:sometab}
\end{table}
Обрано альтернативу 2 як таку, що має на увазі довше життя проекту.


