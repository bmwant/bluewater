\section{Список термінів, скорочень та позначень}

\textit{Прогностична модель} – це набір математичних методів, що використовують статистику для передбачення майбутніх значень досліджуваної величини. 

\textit{Лінійна регресія} – підхід в статистиці для побудови зв’язків між скалярною залежною величиною та однією чи більше додаткових незалежних величин.

\textit{Дата майнінг} (\textit{data mining}) - процес збору даних, а також методи для виявлення закономірностей та обробки даних, що використовується в рамках машинного навчання, статистики та системах для роботи з базами даних. Основною метою цього процесу є отримання корисної інформації з даних та конвертація в зручний уніфікований формат, придатний для подальшого використання.

\textit{Датесет} (\textit{dataset}) - колекція даних. В загальному розумінні відповідає записам в таблиці бази даних. У більш широкому представлені набуває форми матриці, де кожен стовпчик відповідає за деяку змінну, а кожен рядок відповідає значенню цієї змінної для конкретного запису (входження). Датасет виступає в ролі вхідних даних для інстурументів для роботи з даними, а також для алгоритмів класифікації. 

\texit{Характеристичне значення} (\textit{feature}) - значення деякої величини, що характеризує запис вхідних даних. Наприклад, вік може бути таким характеристичним значенням для входження анкети людини, якщо ми збираємо інформацію про групу людей. Відповідає стовпцям вхідного датасету та безпосередньо використовується алгоритмами класифікації для побудови моделей.

\texit{Крос-валідація} (\textit{cross-validation}) - метод оцінки точності передбачень моделі. Показує, на скільки результати моделі на тренувальних даних будуть відрізнятися від результатів на реальних даних. Здійснюється за рахунок тренування моделі на одній частині даних, а валідації цієї ж моделі на інших даних. Далі процес виконується в зворотному напрямку $N$ разів в залежності від кількості частин, на які поділені вхідні дані, за рахунок цього метод і отримав назву \textit{"перехресної перевірки"}.
