\documentclass[14pt]{article}
\usepackage{graphicx}
\usepackage{times}
\usepackage{geometry}
\geometry{
	a4paper,
	left=20mm,
	top=20mm,
	right=15mm,
	bottom=10mm,
}
\usepackage[utf8]{inputenc}
\usepackage[english,ukrainian]{babel}

\begin{document}

\title{Класифікація тексту}
\author{Misha Beherksy}

\maketitle

\begin{abstract}
The abstract text goes here.
\end{abstract}

\section{All about}
На відміну від штучно створених мов, наприклад мов програмування чи математичних нотацій,
мови, які ми використовуємо для спілкування, розвивалися з покоління в покоління, постійно
видозмінюючись, а тому досить складно відслідкувати і встановити набір чітких конкретно
визначених правил. Розробка алгоритмів, що дозволяють "розуміти" людські висловлювання
дає змогу покращити велику кількість аспектів взаємодії людини та комп'ютера: передбачення
вводу, розпізнавання тексту, пошук інформації в неструктурованому тексті, переклад з однієї
мови на іншу, аналіз емоційного забарвлення тексту та багато іншого. Створюючи інтерфейси,
що дозволяють людині більш ефективно використовувати комп'ютер, ми прискорюємо
розвиток багатомовного інформаційного суспільства.
\section{Вступ}
Зі стрімким ростом об'єму інформації онлайн, класифікація тексту стала однією з ключових
технік для обробки та впорядкування даних. Галузі застосування є досить широкими:
починаючи від класифікації новин і закінчуючи персоналізованим пошуком відповідно до
потреб користувача. Оскільки побудова власного класифікатору є досить складним та
часозатратним процесом, доцільно розглянути приклади уже існуючих класифікаторів.
Нижче будуть розглянуті особливості  Support Vector Machines (SVMs) класифікатора в контексті класифікації текстів. Метод був запропонований Володимиром Вапником *-* та
має значні переваги над іншими в швидкодії та у відсутності довгого процесу тонкого
налаштування параметрів моделі.

\section{Exploratory data analysis}
Візуалізація для наступних цілей:
* Комунікативна
- представлення даних та ідей
- проінформувати
- підтримати і аргументувати
- вплинути і переконати
* Дослідницька
- вивчити (дослідити) дані
- проаналізувати ситуацію
- визначити наступні кроки
- прийняти рішення стосовно деякого питання

\begin{equation}
    \label{simple_equation}
    \alpha = \sqrt{ \beta }
\end{equation}

\subsection{Класифікація тексту}
Метою класифікації текстів є розподіл документів на групи наперед визначених категорій. *-*


\section{Висновки}
Результати показують, що стабільно показують чудові результати для завдань класифікації
текстів, суттєво перевищуючи показники інших методів.

\end{document}
