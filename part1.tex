\documentclass[14pt]{article}
\usepackage{graphicx}
\usepackage{times}
\usepackage{geometry}
\geometry{
	a4paper,
	left=20mm,
	top=20mm,
	right=15mm,
	bottom=10mm,
}
\usepackage[utf8]{inputenc}
\usepackage[english,ukrainian]{babel}

\begin{document}

\title{Класифікація тексту}
\author{Misha Beherksy}

\maketitle

\begin{abstract}
The abstract text goes here.
\end{abstract}

\section{Вступ}
Зі стрімким ростом об'єму інформації онлайн, класифікація тексту стала однією з ключових
технік для обробки та впорядкування даних. Галузі застосування є досить широкими: 
починаючи від класифікації новин і закінчуючи персоналізованим пошуком відповідно до
потреб користувача. Оскільки побудова власного класифікатору є досить складним та
часозатратним процесом, доцільно розглянути приклади уже існуючих класифікаторів.
Нижче будуть розглянуті особливості  Support Vector Machines (SVMs) класифікатора в контексті класифікації текстів. Метод був запропонований Володимиром Вапником *-* та 
має значні переваги над іншими в швидкодії та у відсутності довгого процесу тонкого 
налаштування параметрів моделі.

\begin{equation}
    \label{simple_equation}
    \alpha = \sqrt{ \beta }
\end{equation}

\subsection{Класифікація тексту}
Метою класифікації текстів є розподіл документів на групи наперед визначених категорій. *-*


\section{Висновки}
Результати показують, що стабільно показують чудові результати для завдань класифікації
текстів, суттєво перевищуючи показники інших методів.

\end{document}